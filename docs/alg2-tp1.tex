\documentclass[12pt]{article}
\usepackage[utf8]{inputenc}
\usepackage[brazilian]{babel}
\usepackage{hyperref}
\usepackage{lipsum}
\usepackage{geometry}
\usepackage[skip=5pt plus1pt, indent=20pt]{parskip}
\usepackage{indentfirst}
\usepackage{minted}

\usepackage{amsthm,amssymb,amsmath}


\newtheorem*{theorem}{Theorem}

\newcommand{\NN}{\mathbb{N}}
\newcommand{\ZZ}{\mathbb{Z}}
\newcommand{\RR}{\mathbb{R}}
\newcommand{\QQ}{\mathbb{Q}}
\newcommand{\CC}{\mathbb{C}}

\geometry{left=3cm, top=3cm, right=2cm, bottom=2cm}

\title{\textbf{Algoritmos 2 - Trabalho Prático 1 - Manipulação de sequências}}
\author{\textbf{Matheus Flávio Gonçalves Silva - 2020006850}}
\date{\parbox{\linewidth}{\centering%
    Universidade Federal de Minas Gerais (UFMG)\endgraf
    Belo Horizonte - MG - Brasil\endgraf\bigskip
    \href{mailto:matheusfgs@ufmg.br}{matheusfgs@ufmg.br}}}

\begin{document}

\maketitle
%%%%%%%%%%%%%%%%%%%%%%%%%%%%%%%%%%%%%%%%%%%%%%%%%%%%%%%%%%%%%%%%%%%%%%%%%%%%%%

\section{Introdução}
\par O problema proposto consiste na implementação do algoritmo de compressão LZ78 com a utilização de
um dicionário e de uma árvore trie, que não precisa ser compacta, para armazenar os prefixos.

\par O trabalho consiste na implementação da função de compressão dos dados de um arquivo .txt
qualquer de entrada e da descompressão dos dados compactados em um arquivo qualquer de saída. Todos
os outputs possuem um nome padrão que é o nome original do arquivo de entrada, mas com a devida extensão,
mas é possível alterar o output por meio da passagem de um argumento para execução do programa.

%%%%%%%%%%%%%%%%%%%%%%%%%%%%%%%%%%%%%%%%%%%%%%%%%%%%%%%%%%%%%%%%%%%%%%%%%%%%%%
\section{Instruções de Compilação e execução}
\par A compilação e execução do programa é feita de acordo com o indicado na documentação de especificação
do trabalho.

\par \textbf{Compilação:}

\begin{itemize}
	\item Abra um terminal e acesse a pasta raiz do trabalho;
	\item Execute o comando
	      \begin{minted}[
		]{bash}
	make
	\end{minted}
	      no terminal;
	\item Com esse comando, são criados alguns \textbf{".o"} intermediários e o arquivo
	      \textbf{"tp01"} que é o programa compilado e pronto para execução.
\end{itemize}

\par \textbf{Execução:}
\begin{itemize}
	\item Uma vez com o projeto compilado, abra um terminal e acesse a pasta raiz do trabalho;
	\item Execute o programa com um caso teste com o comando:
	      \begin{minted}[
				]{bash}
	./tp01 -option "arquivo_entrada" -o "arquivo_saida"
	\end{minted}

	      sem aspas, em que \textbf{'-option'} é ou \textbf{"-c"} ou \textbf{"-x"} para compressão/descompressão, respectivamente. \textbf{'arquivo\_entrada'} se refere à localização/nome de um arquivo de entrada, tanto para compressão quanto para descompressão. A última parte do comando é opicional, e serve para definir o nome do arquivo de saída tanto para compressão quanto para descompressão;
\end{itemize}

%%%%%%%%%%%%%%%%%%%%%%%%%%%%%%%%%%%%%%%%%%%%%%%%%%%%%%%%%%%%%%%%%%%%%%%%%%%%%%

\section{Implementação}

\par O programa, bibliotecas e cabeçalhos utilizados foram desenvolvidos na linguagem C++,
sendo utilizado o compilador G++ (com padrão de execução conforme dito na documentação do trabalho, a compilação
é dada por meio do comando).
\par O desenvolvimento se deu em um notebook Asus X555UB-BRA-XX299T com processador Core i5 6200u e 8GB de RAM 1333Mhz
LDDR3 com SSD de 240GB SanDisk em ambiente Linux Mint 21.1. Foi utilizado o Visual Studio Code como IDE para
desenvolvimento.

\subsection{Estruturas de Dados}
\par Para a implementação do trabalho foram usados os \textbf{Tipos Abstratos de Dados (TADs)} \textbf{"Node"} e \textbf{"Trie"} para a criação da árvore de prefixos.
\par Cada uma possui os seguintes atributos:
\par \textbf{Node:}
\begin{minted}[]{c++}
char: caracter;
int: code;
std::unordered_map<char, Node*> children; // Armazena os filhos de determinado nó
    \end{minted}
\par Cada um dos atributos é utilizado para criar um elemento da árvore trie.
\par \textbf{Trie:}
\begin{minted}[]{c++}
Node* root;
int nextCode;
int insert(std::string word);// Insere uma nova palavra na árvore
int find(std::string word);// Procura se a palavra passada se encontra na árvore
\end{minted}

\subsection{Funções e Procedimentos}
\subsubsection{Arquivo node.cpp}
\begin{itemize}
	\item \textbf{Node(char caracter, int code)}: cria um novo nó
\end{itemize}

\subsubsection{Arquivo trie.cpp}
\begin{itemize}
	\item \textbf{int Trie::insert(std::string word)}: Insere uma palvra à árvore. Sempre procura se os prefixos que formam a palavra já se encontram. Adiciona apenas o que previamente não estava na árvore.
	\item \textbf{int Trie::find(std::string word)}: Procura por uma palavra na árvore. Retorna código -1 caso a palavra não seja encontrada e retorna o código do nó caso seja o último nó da palavra.

\end{itemize}

\subsubsection{Arquivo lz78.cpp}
\begin{itemize}
	\item \textbf{void compression(string filename, string output)}: Realiza a operação de compressão do texto passado como argumento.
	      Utiliza ifstream e ofstream para leitura e escrita de arquivos. Faz uso de diversas funções padrões e da funções definidas anteriormente nos outros arquivos.
	      Além disso, faz uso de variáveis de auxílio.
	\item \textbf{void decompression(string filename, string output)}: Realiza a operação de descompressão do texto passado como argumento.
	      Utiliza ifstream e ofstream para leitura e escrita de arquivos. Faz uso de diversas funções padrões e da funções definidas anteriormente nos outros arquivos.
	      Além disso, faz uso de variáveis de auxílio.
\end{itemize}


\subsubsection{Arquivo main.cpp}
\begin{itemize}
	\item \textbf{opcaoInvalida()}: Imprime no console que foi passada uma configuração errada de entrada.
	\item \textbf{void compressionReport(string inputName, std::pair<double,double>values)}: Imprime no console que
	os valores teóricos de compressão segundo o método do LZ78. Não reflete o resultado real com tamanho de arquivos.
\end{itemize}
\par O programa principal cria uma série de variáveis de auxílio para a execução do algoritmo.
É o responsável por definir qual operação será realizada e quais os nomes dos arquivos que serão manipulados.
Além disso, imprime no terminal as informações a respeito da compressão dos arquivos, considerando que os bits têm tamanho 8 e que o tamanho da compressão é dado pela dimensão da árvore.

%%%%%%%%%%%%%%%%%%%%%%%%%%%%%%%%%%%%%%%%%%%%%%%%%%%%%%%%%%%%%%%%%%%%%%%%%%%%%%


\section{Análise de Complexidade}
\par A complexidade média estimada da criação da Trie é O(M * N) para a compressão, em que M é o número de
palavras do arquivo e N é o comprimento médio das palavras. Como os textos são textos reais, espera-se que 
esse tamanho médio em geral se aproxime à média de tamanho das palavras mais comumente utilizadas na 
língua em que foi escrito.. A complexidade do programa de descompressão varia linearmente com o
tamanho do arquivo, dado que a reconstrução a partir do dicionário é feita com um vetor, de modo que inserções e consultas
são em O(1), de modo que o tamanho do arquivo domina assintoticamente a operação pela repetição.

%%%%%%%%%%%%%%%%%%%%%%%%%%%%%%%%%%%%%%%%%%%%%%%%%%%%%%%%%%%%%%%%%%%%%%%%%%%%%%
\section{Análise de compressão}
\subsection{constituicao1988.txt:}
\space \par O arquivo original tem tamanho em bits: 5214320
\space \par O arquivo comprimido tem tamanho em bits: 711088
\space \par A taxa de compressão é de 13.6372 \% 

\subsection{dom\_casmurro.txt:}
\space \par O arquivo original tem tamanho em bits: 3276880
\space \par O arquivo comprimido tem tamanho em bits: 592576
\space \par A taxa de compressão é de  18.0835\% 

\subsection{os\_lusiadas.txt:} 
\space \par O arquivo original tem tamanho em bits: 2756304
\space \par O arquivo comprimido tem tamanho em bits: 519544
\space \par A taxa de compressão é de 18.8493\%

\subsection{1-LoremIpsum.txt:} 
\space \par O arquivo original tem tamanho em bits: 8160608
\space \par O arquivo comprimido tem tamanho em bits: 909440
\space \par A taxa de compressão é de 11.1443\%

\subsection{2-CatchingFire.txt:} 
\space \par O arquivo original tem tamanho em bits: 8160608
\space \par O arquivo comprimido tem tamanho em bits: 909440
\space \par A taxa de compressão é de 11.1443\%

\subsection{3-AMetamorfose.txt}                            
\space \par O arquivo original tem tamanho em bits: 966328
\space \par O arquivo comprimido tem tamanho em bits: 191400
\space \par A taxa de compressão é de 19.8069\%

\subsection{4-PoemaCarlosDrummond.txt}
\space \par O arquivo original tem tamanho em bits: 43592
\space \par O arquivo comprimido tem tamanho em bits: 13352
\space \par A taxa de compressão é de 30.6295\%

\subsection{	5-ADivinaComedia.txt}
\space \par O arquivo original tem tamanho em bits: 5923112
\space \par O arquivo comprimido tem tamanho em bits: 1.01318e+06
\space \par A taxa de compressão é de 17.1055%

\subsection{	6-MemoriasPostumasDeBrasCubas.txt            }                                 
\space \par O arquivo original tem tamanho em bits: 2979296
\space \par O arquivo comprimido tem tamanho em bits: 538456
\space \par A taxa de compressão é de 18.0733%

\subsection{	7-OCortico.txt                  }        
\space \par O arquivo original tem tamanho em bits: 4016248
\space \par O arquivo comprimido tem tamanho em bits: 709232
\space \par A taxa de compressão é de 17.6591%

\subsection{	8-TristeFimDePolicarpoQuaresma.txt}                                              
\space \par O arquivo original tem tamanho em bits: 3253912
\space \par O arquivo comprimido tem tamanho em bits: 601416
\space \par A taxa de compressão é de 18.4829%

\subsection{	9-OMulato.txt}                         
\space \par O arquivo original tem tamanho em bits: 4662848
\space \par O arquivo comprimido tem tamanho em bits: 801736
\space \par A taxa de compressão é de 17.1941%

\subsection{	10-PequenoPrincipe.txt}                                  
\space \par O arquivo original tem tamanho em bits: 631584
\space \par O arquivo comprimido tem tamanho em bits: 130272
\space \par A taxa de compressão é de 20.6262%

%%%%%%%%%%%%%%%%%%%%%%%%%%%%%%%%%%%%%%%%%%%%%%%%%%%%%%%%%%%%%%%%%%%%%%%%%%%%%%


\section{Conclusão}
\par A compreensão e desenvolvimento do trabalho em si se deu com um pouco de dificuldade, sendo necessário extrapolar um pouco do horário para finalizar o relatório.
\par Além diss, em tamanho real, não teórico, o único arquivo em que a compressão se deu menor que o texto original foi o LoremIpsum. Creio que isso se deve ao fato de ser um texto com palavras de maior tamanho, de modo que o arquivo .z78 gerado conseguiu reduzir essa média.


%%%%%%%%%%%%%%%%%%%%%%%%%%%%%%%%%%%%%%%%%%%%%%%%%%%%%%%%%%%%%%%%%%%%%%%%%%%%%%%

\section*{Referências}

\par \textbf{VIMIEIRO, Renato} Slides virtuais da disciplina de Algoritmos 2 (2023).
\par Disponibilizado via Teams. Departamento de Ciência da Computação. Universidade Federal de Minas Gerais. Belo Horizonte. Acesso diversas vezes.
\par \textbf{FAROOQ, Omar}"Std Filesystem C++"
\par Disponível em: \textless https://linuxhint.com/std-filesystem-cpp/. Acesso em 9 mai. 2023
\par \textbf{Wikipedia} "LZ78"
\par Disponível em: \textless https://pt.wikipedia.org/wiki/LZ78. Acesso em 9 mai. 2023


\end{document}